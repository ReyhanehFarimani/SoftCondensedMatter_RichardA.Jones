\documentclass[a4paper,12pt]{report}
\usepackage[english]{babel}
\usepackage{microtype}
\usepackage{graphicx}
\usepackage{wrapfig}
\usepackage{enumitem}
\usepackage{fancyhdr}
\usepackage{amsmath}
\usepackage{index}
\makeindex
\title{\large{\textbf{Chapter 3 Soft condensed matter}}}
\author{Reyhaneh Afghahi Farimani\\ 99204008}
\date{March 5, 2021}
\begin{document}
\maketitle
\textbf{3.1.} the phase behavior of a certain liquid mixture can be described by the regular solution model,
 with the interaction parameter being given by $\chi = \frac{600}{T}$,
  where T is the temperature in kelvin. Calculate the following quantities:

\textbf{a)} The temperature at the critical point.

\includegraphics[scale=0.4]{"1.jpg"}
\[
\chi = \frac{600}{T} = 2 \Rightarrow T =300
\]

\textbf{b)} The volume fraction of the coexisting composition at 273k.

\includegraphics[scale=0.4]{"2.jpg"}

\[\phi = \phi_0 , \phi_1 , \phi_2\quad 
\Rightarrow \frac{\partial \Delta F}{\partial \phi} = 0\]
\[
\Delta F = \phi \ln(\phi) +(1- \phi)\ln(1- \phi) + \chi \phi (1- \phi)
\]
\[
\frac{\partial \Delta F}{\partial \phi} = 0 
\Rightarrow \ln(\frac{\phi}{1-\phi}) + \chi (1 - 2\phi) =0  
\Rightarrow \ln(\frac{\phi}{1-\phi}) + \frac{600}{273} (1 - 2\phi) =0 
\]
\[
\phi_1 =  0.24969272763784 \quad \phi_2 = 0.750307272362156
\]
\textbf{c)}The volume fraction at the spinodal line at 273K.
\[
\frac{\partial^2 \Delta F}{\partial \phi^2} = 0 
\Rightarrow \frac{1}{\phi} + \frac{1}{1- \phi} -2\chi =0 
\Rightarrow \frac{1}{\phi} + \frac{1}{1- \phi} = \frac{1200}{273}
\]
\[
\phi_a = \frac{7}{20} =0.35 ,\quad \phi_b = \frac{13}{20} =0.65
\]

\textbf{3.2.a)}
Use regular solution theory to drive an approximation expression,
 valid in the limit of a large, positive value of interaction parameter $\chi$,
 for the limit of solubility of one immiscible liquid in another.

\[
 \frac{\partial \Delta F}{\partial \phi} = 0 
 \to ln(\frac{\phi}{1 - \phi}) + \chi (1 - 2 \phi) =0
\]
\[
\chi \to \infty \Rightarrow \phi \to 0 
\Rightarrow ln(\phi) +\chi =0 
\]
\[
\phi = e^{-\chi}
\]
\textbf{b)}Comment on your result.

since $\chi \propto \frac{U}{k_BT}$, $e^{-\chi}$ is simply Boltzmann factor.

\textbf{3.3.} The value of $\chi$ between water and linear hydrocarbons may 
be taken to be given by the formula $\chi = 3.04 + 1.37n_C$, 
where $n_C$ is the number of carbon atoms in the hydrocarbon. 
Use the formula derived in the last question to compare the 
limiting solubility in water of hexane ($C_6H_{14}$),
octane ($C_8H_{18}$), and decane ($C_{10}H_{22}$).

hexane: \[
\chi = 3.04 + 1.37\times 6 = 11.26\Rightarrow \phi =
 e^{-11.26} = 0.000012878 =1.2878 \times 10^{-5}
\]

octane: \[
\chi = 3.04 + 1.37\times 8 = 14\Rightarrow \phi =
 e^{-14} = 0.000000832 = 8.32 \times 10^{-7}
\]

decane: \[
\chi = 3.04 + 1.37\times 10 = 16.74
\Rightarrow \phi = e^{-16.74} = 0.000000054 = 5.4\times 10^{-8}
\]

\textbf{3.4.}Estimate the inter facial tension between octane and water.
 You may take the interaction parameter between octane and 
 water as $\chi=14.0$ and the molecule volume as $2.36 \times 10^{-29} m^3$.

\[
\gamma =\frac{\chi k_B T}{z v^{\frac{2}{3}}} = 
\frac{14.0 \times 4.11 \times 10^{-21}J}{8 \times 2.36^{\frac{2}{3}} \times 10^{-29 \times \frac{2}{3} }m^2} = 0.087920325 [\frac{J}{m^2}] = 0.087920325 [\frac{N}{m^3}]
\]

\textbf{3.5.} A light scattering experiment is carried out on a phase separating polymer mixture.
 Values of intensity are recorded as a function of time at a variety of scattering angles q.
  The data is shown.

\textbf{a)} Plot the scattered intensity as a function time for $ Q = 1.21 \mu m^{-1}$.
 Explain the shape of the curve.

\includegraphics[scale=0.6]{"1.21.jpg"}

according to Cahn Hillard equation we expect:
\[
\phi(x,t) \propto \exp(R(Q,t)t)
\]
\[
I(Q,t) \propto \exp(2R(Q,t)) \to \ln(I(Q,t)) =2R(Q,t)
\]
Lifshitz Slyozov law:
\[
R(Q,t) \propto t^{\frac{1}{3}} \Rightarrow ln(I(Q,t)) \propto t^{\frac{1}{3}}
\]

fitting using python scipy package:

\includegraphics[scale=0.6]{"fit.jpg"}

\textbf{b)} Use the data to extract values of the amplification factor,$R(Q)$,
 as a function of scattering wavefactor?


fitted value of R(q) in scipy:

\[
R(Q) = 2.4624513 
\]

\textbf{c)}
How according to Cahn Hillard theory do you expect R(Q) to vary with Q?
Plot your values of on the degree of argument and comment on the degree of agreement?

according to Cahn Hillard equation we expect:
\[
I(Q,t) \propto \exp(2R(Q)) \to \ln(I(Q,t)) =2R(Q)
\]
\[
R(q) = -D_{eff} q^2 (1+\frac{2\kappa q^2}{f"_0})
\]
we do what we have done for Q=1.21, 
for other values of Q as well and obtain R(Q) as a function of Q.
plotting using python matplotlib package:

\includegraphics[scale=0.3]{"0.97.jpg"}
\includegraphics[scale=0.3]{"1.21.jpg"}
\includegraphics[scale=0.3]{"1.46.jpg"}

\includegraphics[scale=0.3]{"1.86.jpg"}
\includegraphics[scale=0.3]{"2.78.jpg"}
\includegraphics[scale=0.3]{"3.jpg"}

plotting R(Q) versus Q:

\includegraphics[scale=0.6]{"fit2.jpg"}

fitting this function, using python scipy package:

\[
R(q) = -D_{eff} q^2 (1+\frac{2\kappa q^2}{f"_0})
\]
\includegraphics[scale=0.6]{"fit3.jpg"}

\textbf{d)} Use the graph to estimate a value of the effective diffusion coefficient $D_{eff}$.
\[
D_{eff} =-0.20581287 \to f"<0
\]

\textbf{3.6}Droplet of molten silver, with radius $100\mu m$,
 are observed under a microscope as the temperature is lowered below the melting point.
  A large number of droplets all solidify 227K below the melting point.

\textbf{a)} Assuming that these droplets are solidifying by homogeneous nucleation,
 calculate the solid liquid inter facial energy of silver.
  You may assume that the droplet solidifies when it contains,
  one nucleus of the critical size,
  and that during the experimental time scale, 
  each atom makes $5\times10^{13}$ attempts to form a nucleus.
\[
\exp(\frac{-\Delta G^*}{k_BT}) = \frac{1}{Number Of Attemps} =
 \frac{1}{5\times 10^{13}\times N}
\]
\[
N = V_{droplet} [cm^3] \times Density[gcm^{-3}] \times AtomicMass[g N_{A}^{-1}] \times N_{A}
\]
\[
N = \frac{4}{3} \pi (10^{-2})^3 \times 10.49 \times 108 \times 6.63 \times10^{23} = 3
.1 \times 10^{21}
\]
\[
\frac{\Delta G^*}{k_BT} = \ln(5\times 10^{13}\times3.1 \times 10^{21})=
 34\ln{10} + \ln{5} + \ln{3.1} = 81.028733186
\]
equation 3.36 in book:
\[
\Delta G^* = \frac{16\pi}{3} \gamma_{sl}^{3} (\frac{T_m}{\Delta H_m})^2 \frac{1}{\Delta T^2}
\]
\[
81.028 \times (1234 - 227) \times 1.38 10^{-23} J =
 \frac{16\pi}{3} \gamma_{sl}^{3} (\frac{1234K}{ 1.1\times 10^9 Jm^{-3}})^2 \frac{1}{ 227^2 K^2}
\]
\[
\gamma_{sl} = 0.140131334[Jm^{-2}]
\]

\textbf{b)} Recent computer simulation suggest that the interface between a crystal and its melt,
 rather than being automaticly sharp,
  is between 5 and 10 atomic spacing broad.
  What the implification of this is result for classical nucleation theory?
  Illustrating your answer by calculating the classical critical nucleus size in part a.

It would reduce the value of $\gamma_{sl}$.
\[
r^* = \frac{2\gamma_{sl} T_m}{\Delta H_m \Delta T} = 
\frac{2\times 0.14013 Jm^{-2} \times 1234 K}{1.1\times 10^9 Jm^{-3} 277K} = 
1.13 nm 
\]

\end{document}
