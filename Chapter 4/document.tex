\documentclass[a4paper,12pt]{report}
\usepackage[english]{babel}
\usepackage{microtype}
\usepackage{graphicx}
\usepackage{fancyhdr}
\usepackage{amsmath}
\usepackage{index}
\makeindex
\title{\large{\textbf{Chapter 4 Soft condensed matter}}}
\author{Reyhaneh Afghahi Farimani\\ 99204008}
\date{March 14,2021}
\begin{document}
\maketitle
\textbf{(4.1)}
For each of the following particles, suspended in water,\\ a) a grain of sand, $100 \mu m$ in diameter, density $2200 kgm^{-3}$,\\ b) a polymer particle, $1 \mu m$ in diameter, density $1050 kgm^{-3}$,\\ c) a virus particle, $50 nm$ in diameter, density $1020 kgm^{-3}$,\\i.calculate the terminal velocity.
\[
F = \zeta v = 6\pi \eta a v
\]
\[
F = m^* g = V \delta \rho g = \frac{4}{3} \pi a^3 \delta \rho g
\]
\[
\Rightarrow \frac{4}{3} \pi a^3 \delta \rho g = 6\pi \eta a v
\]
\[
v = \frac{2a^2\delta \rho g}{9 \eta }=\frac{[m^2][kg m^{-3}][ m s^{-2}]}{[kg m^{-1} s^{-1}]}=[ms^{-1}]
\]
\[
v = \frac{20a^2\delta \rho }{9 \times 1.002 \times 10^{-3} }= Ca^2\delta \rho ,\quad C = 2217.786
\]
\[
v_a = C (100\mu m)^2 (2200 - 1000) = 2.661 \times 10^{-2} ms^{-1}
\]
\[
v_b = C (1\mu m)^2 (1050 - 1000) = 1.108\times 10^{-7} ms^{-1}
\]
\[
v_c = C (50n m)^2 (1020 - 1000) = 1.108\times 10^{-10} ms^{-1}
\]
ii.calculate the diffusion constant.
\[
T = T_{room}
\]
\[
D = \frac{k_BT}{\zeta} =\frac{[J][ms^{-1}]}{[N]} =[m^2s^{-1}]
\]
\[
D = \frac{k_BT}{\zeta} = \frac{k_BT}{6\pi \eta a} = \frac{4.11\times 10^{-21}}{6\pi \times 1.002 \times 10^{-3} a} 
\]
\[
D = 2.17\times 10^{-19} \frac{1}{a}
\]
\[
D_a = 2.17\times 10^{-19} \frac{1}{100 \mu m} = 2.17 \times 10^{-15}m^2s^{-1}
\]
\[
D_b = 2.17\times 10^{-19} \frac{1}{1 \mu m} = 2.17 \times 10^{-13}m^2s^{-1}
\]
\[
D_c = 2.17\times 10^{-19} \frac{1}{50 nm} = 4.35 \times 10^{-12}m^2s^{-1}
\]
iii.estimate the time taken for the particle to diffuse a distance equal to its own diameter.
\[
<R^2> = 6Dt \to t=\frac{<R^2>}{6D}
\]
\[
t_a = \frac{a_a^2}{6D_a}=\frac{(100 \mu m)^2}{6\times 2.17 \times 10^{-15}m^2s^{-1}} =7.68 \times 10^5 s
\]
\[
t_b = \frac{a_b^2}{6D_b} = \frac{(1 \mu m)^2}{6\times 2.17 \times 10^{-13}m^2s^{-1}}=7.68 \times 10^{-1} s
\]
\[
t_c = \frac{a_c^2}{6D_c}=\frac{(50 nm)^2}{6\times 4.35 \times 10^{-12}m^2s^{-1}}= 9.578\times 10^{-5}s
\]


\textbf{(4.2)} Consider a colloid consisting of particles of mass $m$ and density $\rho_c$ suspended in water.\\
a.Show that the number density as a function of the height from the bottom of the container, $z$ ,$n(z)$, is given by,
\[
n(z) = n_0\exp(-\frac{m(\rho_c - \rho_w)gz}{\rho_ck_BT})
\]

Boltzman equation:
\[
n(z) = n_0 \exp(-\frac{U(z)}{k_BT})
\]
\[
U(z) = m*gz = V\delta \rho gz = \frac{m\delta \rho g}{\rho}= \frac{m(\rho_c - \rho_w)gz}{\rho_c}
\]
\[
\Rightarrow n(z) = n_0\exp(-\frac{m(\rho_c - \rho_w)gz}{\rho_ck_BT}) 
\]
b.consider a slice of the suspension between heights $z$ and $z+dz$.\\i.show that diffusion leads to a net flux of particles $J_D$ into the slice given by
\[
J_D = D\frac{m^2 g^2}{k_{B}^2T^2}\frac{(\rho_c - \rho_w)^2}{\rho_{c}^2}n(z)dz
\]

\[
J_D =jdz = \frac{I}{A}dz = \frac{\partial n}{\partial t}dz
\]
Fick's second law:
\[
\frac{\partial n}{\partial t} = D\frac{\partial^2 n}{\partial z^2}
\]
\[
\frac{\partial n}{\partial z}=-n_0\frac{m(\rho_c - \rho_w)g}{\rho_ck_BT}\exp(-\frac{m(\rho_c - \rho_w)gz}{\rho_ck_BT})
\]
\[
\frac{\partial^2 n}{\partial z^2} = n_0(\frac{m(\rho_c - \rho_w)g}{\rho_ck_BT})^2\exp(-\frac{m(\rho_c - \rho_w)gz}{\rho_ck_BT}) = (\frac{m(\rho_c - \rho_w)g}{\rho_ck_BT})^2 n(z)
\]
\[
J_D = D (\frac{m(\rho_c - \rho_w)g}{\rho_ck_BT})^2 n(z) dz = D\frac{m^2 g^2}{k_{B}^2T^2}\frac{(\rho_c - \rho_w)^2}{\rho_{c}^2}n(z)dz
\]
ii.Show that the sedimentation under gravity leads to a net flux of particles out of the slice $J_s$ given by
\[
J_s = \frac{m^2 g^2}{\zeta k_{B}T}\frac{(\rho_c - \rho_w)^2}{\rho_{c}^2}n(z)dz
\]

\[
J_s = jdz = \frac{I}{A}dz = \frac{dn}{dt}dz = \frac{\partial n}{\partial z}\frac{dz}{dt}dz = \frac{\partial n}{\partial z}v_zdz
\]
\[
v_z=\frac{F_{gravity}}{\zeta} = \frac{m^*g}{\zeta} =\frac{m g(\rho_c - \rho_w)}{\zeta \rho_{c}}
\]
\[
\frac{\partial n}{\partial z} = -n_0\frac{m(\rho_c - \rho_w)g}{\rho_ck_BT}\exp(-\frac{m(\rho_c - \rho_w)gz}{\rho_ck_BT})
\]
\[
J_s = -n_0\frac{m(\rho_c - \rho_w)g}{\rho_ck_BT}\exp(-\frac{m(\rho_c - \rho_w)gz}{\rho_ck_BT}) \frac{m g(\rho_c - \rho_w)}{\zeta \rho_{c}}
\]
\[
J_s = -\frac{m^2 g^2}{\zeta k_{B}T}\frac{(\rho_c - \rho_w)^2}{\rho_{c}^2}n(z)dz
\]
net flux of particles \textbf{out of} the slice:
\[
\Rightarrow J_s = \frac{m^2 g^2}{\zeta k_{B}T}\frac{(\rho_c - \rho_w)^2}{\rho_{c}^2}n(z)dz
\]
iii.Use the Stokes Einstein law to show that the flux into the slice due to diffusion is balanced by the flux out of the slice due to sedimentation.
\\
Stokes Einstein law:
\[
D = \frac{k_BT}{\zeta}
\]
\[
J_D = D\frac{m^2 g^2}{k_{B}^2T^2}\frac{(\rho_c - \rho_w)^2}{\rho_{c}^2}n(z)dz = \frac{m^2 g^2}{\zeta k_{B}T}\frac{(\rho_c - \rho_w)^2}{\rho_{c}^2}n(z)dz = J_s
\]
\[
flux\_into\_the\_slice = flux\_out\_of\_the\_slice
\]

\textbf{(4.3)}Consider the van der Waals interaction between two semi-infinite gold plates interacting across a vacuum.\\
a. Compare the force per unit area as predicated by the Hamaker approach, taking the value of Hamaker constant $A = 2\times 10^{-19} J$, with the value of the Casimir force,for values of the plates separation of\\i.$1 \mu m$\\ii.$100 nm$\\iii.$1 nm$.
\\van der Waals:
\[
U_{vdW}(h) = -\frac{A}{12\pi h^2} = -\frac{5.305\times 10^{-21}}{h^2} 
\]
\[
U_{vdW}(1 \mu m) = -\frac{5.305\times 10^{-21}}{(1 \mu m)^2} =-5.305\times 10^{-9}J
\]
\[
U_{vdW}(100 nm) = -\frac{5.305\times 10^{-21}}{(100 nm)^2} =-5.305\times 10^{-7}J
\]
\[
U_{vdW}(1 nm) = -\frac{5.305\times 10^{-21}}{(1 nm)^2} =-5.305\times 10^{-3}J
\]
Energy due to Casimir force:
\[
E_{Casimir}(h) = -\frac{\hbar c \pi^2}{720 h^3}=-\frac{4.336\times 10^{-28}}{h^3}
\] 
\[
E_{Casimir}(1 \mu m) =-\frac{4.336\times 10^{-28}}{(1 \mu m)^3}= -4.336\times 10^{-10}J
\]
\[
E_{Casimir}(100 nm) =-\frac{4.336\times 10^{-28}}{(100 nm)^3}=-4.336\times 10^{-7}J
\]
\[
E_{Casimir}(1 nm) =-\frac{4.336\times 10^{-28}}{(1 nm)^3}=-4.336\times 10^{-1}J
\]
b. At what separation $h_x$ are two forces predicted to be equal?
\[
U_{vdW}(h_x) = E_{Casimir}(h_x)
\]
\[
\frac{5.305\times 10^{-21}}{h_x^2}=\frac{4.336\times 10^{-28}}{h_x^3}
\]
\[
h_x = \frac{4.336\times 10^{-28}}{5.305\times 10^{-21}}m=\frac{4.336\times 10^{-7}}{5.305}m = 8.162\times 10^{-8}=81.62 nm
\]
c.Which expression is more likely to be accurate for separation grater than $h_x$, and which is more accurate less than $h_x$? Give reason for your answer in each case.\\
$h<h_x$:
Casimir is not accurate enough, because real metals are not perfect conductor and the field would penetrate a certain depth, of order 100nm.\\
$h>h_x$:
Van der Waals force is not accurate enough as it does not consider finite speed of propagation of electromagnetic field.

\textbf{(4.4)}Consider a colloid of charged spheres all of radius $0.1\mu m$ in an aqueous solution of sodium chloride.\\a. Calculate the Debye screening length for salt concentration of $10^{-5}, 10^{-4}, 10^{-3}, 10^{-2}$.
\[
\kappa^{-1} = (\frac{K\epsilon_0 k_BT}{e^2Cz^2})^{\frac{1}{2}} =  (\frac{80 (8.85 \times 10^{-12}) (4.11 \times 10^{-21})}{(1.6 \times 10^{-19})^2C'\times 6.63\times 10^{23}\times 2})^{\frac{1}{2}} = \frac{6.54\times 10^{-9}}{\sqrt{C'}}
\]
\[
\kappa^{-1}_{C=10^{-5}} = \frac{6.54\times 10^{-9}}{\sqrt{10^{-5}}} = 2.07\times 10^{-6}m =2.92 \mu m
\]
\[
\kappa^{-1}_{C=10^{-4}} = \frac{6.54\times 10^{-9}}{\sqrt{10^{-4}}} = 6.54\times 10^{-7}m = 925 n m
\]
\[
\kappa^{-1}_{C=10^{-3}} = \frac{6.54\times 10^{-9}}{\sqrt{10^{-3}}} = 2.07\times 10^{-7}m = 292 n m
\]
\[
\kappa^{-1}_{C=10^{-2}} = \frac{6.54\times 10^{-9}}{\sqrt{10^{-2}}} = 6.54\times 10^{-8}m = 92.5 n m
\]\\b. For each of the salt concentration above, estimate the volume fraction for the transition to an ordered phase. You may assume that the particles may be considered to behave as hard spheres with an effective radius equal to the sum of the physical radius and the Debye screening length.
\[
\frac{n_cr_c^3}{n_sr_s^3}= \phi,\quad \frac{n_c(r_c+\kappa ^{-1})^3}{n_sr_s^3}= \phi_0\Rightarrow \phi [\frac{r_c+\kappa^{-1}}{r_c} ]^3 = \phi_0 =0.67
\]
\[
\phi[1+\frac{\kappa ^-1}{r_c}]^3 = 0.67 \Rightarrow \phi=0.67[1+\frac{\kappa ^{-1}}{r_c}]^{-3}
\]
\[
\phi_{C=10^{-5}}=0.67[1+\frac{\kappa ^{-1}_{C=10^{-5}}}{r_c}]^{-3} =0.67[1+\frac{2.92 \mu m}{0.1 \mu m}]^{-3}=2.43\times 10^{-5}
\]
\[
\phi_{C=10^{-4}}=0.67[1+\frac{\kappa ^{-1}_{C=10^{-4}}}{r_c}]^{-3} =0.67[1+\frac{925 n m}{0.1 \mu m}]^{-3}=6.22 \times 10^{-4}
\]
\[
\phi_{C=10^{-3}}=0.67[1+\frac{\kappa ^{-1}_{C=10^{-3}}}{r_c}]^{-3} =0.67[1+\frac{292 n m}{0.1 \mu m}]^{-3}=1.11\times 10^{-2}
\]
\[
\phi_{C=10^{-2}}=0.67[1+\frac{\kappa ^{-1}_{C=10^{-2}}}{r_c}]^{-3} =0.67[1+\frac{92.5 n m}{0.1 \mu m}]^{-3} = 9.39 \times 10^{-2}
\]

\textbf{(4.5)}A water-based varnish is composed of a dispersion of polymer spheres with diameters of 200nm.If it is brushed onto a surface as a film of thickness $200\mu m$,how fast must the brush be moved to achieve an appreciable degree of shear thinning?
\[
Pe =1 \to \tau_{shear} = \tau_{diffusion} =\frac{6\pi \eta_{water} a^3}{k_BT} =\frac{6\pi \times 1.002 \times 10^{-3} Pa.s \times (200nm)^3}{4.11\times 10^{-21}J}
\]
\[
v = \frac{L}{\tau_{shear}} = \frac{200\mu m \times 4.11\times 10^{-21}J}{6\pi \times 1.002 \times 10^{-3} Pa.s \times (200nm)^3} = \frac{2 m \times 4.11 J \times 10^{-25}}{6\pi\times  1.002Pa.s\times (2m)^3\times 10^{-24}}
\]
\[
v = \frac{2  \times 4.11  \times 10^{-1}}{6\pi\times  1.002\times 8}\frac{m}{s}=5.44\times10^{-3}\frac{m}{s}
\]
\end{document}
