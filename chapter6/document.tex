\documentclass[a4paper,12pt]{report}
\usepackage[english]{babel}
\usepackage{microtype}
\usepackage{graphicx}
\usepackage{fancyhdr}
\usepackage{amsmath}
\usepackage{index}
\makeindex
\title{\large{\textbf{ Chapter Six Soft condensed matter\\ Richard A.L.Jones}}}
\author{Reyhaneh Afghahi Farimani\\ 99204008}
\date{June 19,2021}
\begin{document}
\maketitle
\textbf{6.1.} In a certain chemical cross-linking reaction involving a monomer that can react at three sites,
 the degree of reaction f obeys the second-order rate law
\begin{equation} 
\frac{df}{dt} = k (1-f)^2
\end{equation}
where the rate constant k has the value $ 4 \times 10^{-4} s^{-1}$. Use the Flory\_Stochmayer  theory to calculate

a. the times at which the gel point is reached,
\[
(1) \to \frac{df}{dt} = k (1-f)^2 \to \frac{df}{(1-f)^2} = kdt
\]
\[
\int_{0}^{f} \frac{df'}{(1-f')^2} = \int_{0}^{t} kdt' \to \frac{1}{1-f} - 1 = kt
\]
\begin{equation} 
	\frac{f}{1-f} = kt
\end{equation}
\[
(6.4) \to f_c = \frac{1}{z-1} = \frac{1}{3-1} = \frac{1}{2}
\]
\[
t_c = \frac{1}{k} (2-1) = \frac{1}{k} = \frac{1}{4} \times 10^4 s = 2500s
\]

b. the time after which three quarters of monomers have been polymerized,

\[
f = \frac{3}{4} , (2) \to 3 = kt \to t = \frac{3}{k} = \frac{3}{4} \times 10^4 s 
\]
\[
t = 7500s
\]

c. the time after which three quarters of monomers form part of the infinite network.

\[
(6.7) \to P = f(1 - Q^3) , f>f_c \to P = f(1 - (\frac{1-f}{f})^3)
\]
\[
\frac{3}{4} = f(1 - (\frac{1-f}{f})^3)
\]
roots:
\[
f_1 =0.77039,f_2 =0.55230 - 0.58650 i, f_3 = 0.55230 + 0.58650 i
\]
\[
(2)\to kt = \frac{0.77039}{1-0.77039}=3.35521101
\]
\[
t = \frac{1}{k}3.35521101  = 8388.027s
\]

\textbf{6.2.}In an experiment to test the application of the theory of peculation to gelation,
 the gel fraction is determined when the fractional extent of reaction is a small degree
  $\Delta f$ larger than its value at the gel point.

a. Is the value of gel fraction at a fractional extent of reaction $\Delta f/2$
 larger or smaller when predicated by peculation theory than
  the value predicted by Flory\_Stochmayer theory?

b.By what factor do the two prediction differ?

we know for classical peculation theory the gel fraction can be expanded:
\[
(6.9) \to \frac{P}{f} = 3(f-f_c) + O(f-f_c)^3
\]
Considering the leading order the relation is linear and therefore:
\[
\Delta f \to \frac{\Delta f}{2} \Rightarrow \frac{P}{f} \to \frac{P}{f}\times0.5
\] 
By contrast, Monte Carlo Simulation of bond peculation in 3D gives:
\[
\frac{P}{f} = (f-f_c)^{0.41}
\]
therefore
\[
\Delta f \to \frac{\Delta f}{2} \Rightarrow \frac{P}{f} \to \frac{P}{f}\times0.7526
\]
the value predicted by Flory\_Stochmayer theory is smaller.

\end{document}
